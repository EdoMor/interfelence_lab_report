\section{Theoretical background }
The nature of flight is a much discussed topic
One view on the matter is that light behaves like a wave as can be seen by the maxwell equations,
one possible solution to the maxwell equations in vacuum is $\vec{E}=\frac{\vec{E_0}}{2}e^{i \left( \vec{k}\cdot\vec{r}-\omega t \right)}+c.c$
and $\vec{B}=\frac{\vec{B_0}}{2}e^{i\left( \vec{k}\cdot\vec{r}-\omega t \right)}+c.c$ where $\|\vec{B_0}\|=\frac{\|\vec{E_0}\|}{c}$ and $\vec{B_0}\bot\vec{E_0}$
which describes a plane wave propagating in the $\vec{k}$ direction.\\
Under this interpretation we expect light to Create an interference pattern when propagating passed a block.\\
The Huygens principle states that each point in a wavefront can be treated as a point wave source,
To define the shape we shall use an optical transference function (\textbf{OTF}) $T(x,y)$
defined to be 1 where the light is completely unobstructed 0 where no light passes threw and can get any complex value depending on the properties of the block
let us also omit the $+c.c$ from now on.\\
For convenience we shall call the axis perpendicular to our block $z$ and the axis in its plane $x$ and $y$ furthermore we shall call the plane where our wave meets the block
the $z=0$ plane, Thus the wave immediately after the block is expressed by $U_o(x,y,0^-)T(x,y)e^{-i\omega t}$ Where $U_o(x,y,z)$ In the spatial part of the wave function
describing the origin illuminating our block.
We shall mark $\psi(x,y,0,t)=T(x,y)U_o(x,y,0^-)e^{-i\omega t}$ As the wavefunction describing our wave immediately after the block
Such a wavefunction can be described in terms of plane waves as $\psi(\vec{r},t)=\left( \frac{1}{\sqrt{2\pi}} \right)^3 \iiint\limits_{-\infty}^{\infty}b(\vec{k})e^{i \left( \vec{k}\cdot\vec{r}-\omega_k t \right)}d^{3}k$
where $\omega_k=\|\vec{k}\|c=k_{0}c \Rightarrow k_z=\sqrt{k_0^2-k_x^2-k_y^2}$
\\Let us now examine the interference pattern created by a plain wave When hitting a block of some shape.\\
