\section{Appendix}
our sensor is at a fixed distance from the slits with $z=0.855 [m]$.
The angle $\theta$ and the Intensity $I$ were measured using a resistor and a photodiod, the voltage on these resistors ($V\propto\theta,I$) was
measured at a rate of $40[Hz]$ and a resolution of $\Delta V\approx10^{-3}[V]$.
Prior to every measurement we align the laser beam with the sensor by detecting the point of maximum intensity, then place the slits perpendicular to
the beam.
The slits are then moved along the x axis until we measure maximum amplitude for a single slits and n slits, or between the two maxima for double slit patterns.
These procedures are key to make sure our wave is a good approximation for a plain wave and it is reaching the slits with uniform phase thus avoiding major differences from our models' assumptions.
The sensor was then moved to scan the intensities at some range of angles which were then converted to distances on our theoretical screen.
As can be seen in figure \ref{fig:single slit interference with 0.04mm width} the theory for a single slit fits the data well however there are
secondary effects easily observable in the tails of the data that per our understanding are partially caused by the non-linearity of the
photomultiplier and photoelectric sensor, such effects were not taken into account when formulating our theory
However the adjustment can be easily accomplished.
\begin{figure}[H]
    \includegraphics[width=0.9\columnwidth]{figures/single slit interference with 0.04mm width.png}
    \caption{a comparison of the interference pattern of a 0.04 mm with slit with the theoretical prediction.\\
    Note the pattern within the tails of the residual graph showing presence of a secondary effect.($\sigma\sim0.0086$)}
    \label{fig:single slit interference with 0.04mm width}
\end{figure}
\begin{figure}[H]
	\centering
	\begin{subfigure}{0.5\columnwidth}
		\centering
		\includegraphics[width=\columnwidth]{figures/single slit interference 0.08mm.png} % first figure itself
		\caption{}
        \label{fig:single slit interference 0.08mm}
	\end{subfigure}\hfill
    \begin{subfigure}{0.5\columnwidth}
        \centering
        \includegraphics[width=\columnwidth]{figures/single slit interference 0.16mm.png} % second figure itself
        \caption{}
        \label{fig:single slit interference 0.16mm}
    \end{subfigure}
    \begin{subfigure}{0.5\columnwidth}
        \centering
        \includegraphics[width=\columnwidth]{figures/single slit interference with 0.04mm not best fit but better tails.png}
        \caption{this figure is used to show with greater clarity the secondary effects in the tails of the interference pattern residuals graph}
        \label{fig:single slit interference with 0.04mm not best fit but better tails}
    \end{subfigure}
    \label{fig:single slit examples}
\end{figure}

