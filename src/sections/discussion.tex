\section{Discussion}
We were able to successfully show in our experiment that using a fairly simplistic model
Very accurate results can be obtained with a typical error of about $0.09\sigma$
However there were some secondary effects that were not all taken into account
Some of these effects were diffraction due to the iris, summing of the intensity due to the width of the iris, correcting $I(x)$ as
$I(x)=\int\limits_{x-d}^{x+d}I(x')dx'$ where d is half the width of the iris(this was taken into account),
memory of the photoelectric sensor and non-linearity in the voltage of the photomultiplier.
furthermore we have shown that for a semi-translucent sample with typical size of $\lambda$ a reconstruction of its shape
is possible given that the distance from the screen is large enough the error in such reconstruction is proportional to the
size of the screen and most of the details lost due to the cutoff will effect the outer parameter of the shape.
