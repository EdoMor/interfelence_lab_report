\section{Results}\label{sec:results}
In order to be able to correctly analyse the diffraction pattern of a helix let us deconstruct it into segments.
Slit patterns can be thought of as an approximation of one-dimensional sections of a helix, they also have a well-known solution.
\subsection{Double-slit patterns}
According to the theory we used, the intensity for double slit patterns can be approximated as:
\begin{aligned}
        I(x)=\frac{kd^2\cos^2\left(\frac{kLx}{2z}\right)sinc^2\left(\frac{dkx}{2z}\right)}{z\pi}
\end{aligned}

\begin{figure}[H]
    \includegraphics[width=0.9\columnwidth]{figures/0.04w0.25s memory.png}
    \caption{The secondary effects caused by measuring method are demonstrated here by the blue dashed line, we never see 0 intensity contrary to theory, and the visible offset in the fit relative to measurements. $(\sigma\sim0.01)$)}
    \label{fig:0.04w0.25s memory}
\end{figure}
After taking into consideration the iris effect, our model describes our measurements with an error of $(\sigma\sim0.01)$
Where $\sigma=$
The local minima seem to increase in value closer to $x=0$, and we are able to predict tha general shape of the diffraction.
We also see an offset between theory and experiment that is, especially around the tails, one-sided.
We believe that tendency towards a certain side is duo to the fact that we scanned in a certain direction and the photoelectric sensor's measurement at a certain point is affected by measurements made a short time before that.


As seen in Figure 6, when increasing the width of each slit we see that the $sinc$ function is narrower, and for greater spaces between slits we see the increase in the $cos$ function's frequency as expected.
What's not predicted by our theory is the decrease in maximum amplitude which shouldn't change when altering the spacing,
that can be explained by the fact we centered the maximum amplitude of the laser between slits for symmetry thus creating a decrease in amplitude of the incoming plane wave.
\begin{figure}[H]
    \centering
    \begin{subfigure}{0.5\columnwidth}
        \centering
        \includegraphics[width=\columnwidth]{figures/0.08w0.25s.png} % first figure itself
        \caption{}
        \label{fig:double slit interference 0.08w0.25s}
    \end{subfigure}\hfill
    \begin{subfigure}{0.5\columnwidth}
        \centering
        \includegraphics[width=\columnwidth]{figures/0.08w0.5s.png} % second figure itself
        \caption{}
        \label{fig:single slit interference 0.16mm}
    \end{subfigure}
    \caption{Diffraction patterns for different slit patterns, for wider slits we get a narrow envelope of main peaks and for greater spacing we see denser peaks.}
\end{figure}
\begin{figure}[H]
    \includegraphics[width=0.9\columnwidth]{figures/10 slits per mm with residuals.png}
    \caption{Interference pattern of a slide containing 10 slits per mm of width 0.04 mm spaced 0.06 mm apart\\
    ($\sigma\sim0.086)$}
    \label{fig:10 slits per mm}
\end{figure}
\newpage
In the case of the periodic diffraction grating "10 lines per mm" the secondary effects are more pronounced especially
the photoelectric sensor's "memory" (The photoelectric sensor has a relaxation period in which to voltage diminishes
therefore after exposure instead of an immediate cut off a slope can be seen as the voltage is recorded with the relation
to the angle which continues to change during said period giving us higher peaks and wider slopes near those peaks)\\
\\
Finally To justify the title of the paper we should look at the interference pattern of the helix (Which bears some resemblance the shape off DNA)
according to equation \eqref{eq:farfield} and the specified approximations the interference pattern can be calculated using
the 2 dimensional fourier transform of the shape of the slit, Taking the inverse fourier transform of the pattern should result
In the square of the shape of the slit (since when calculating the intensity the square of $U$ was taken)
as can be shown by \ref{fig:expansion inverse fourie transform}
\begin{figure}[H]
    \includegraphics[width=0.9\columnwidth]{figures/expantion meshured interferemce.png}
    \caption{interference pattern measured as a result of diffraction with a helix}
    \label{fig:expansion measured interference pattern}
\end{figure}
\begin{figure}[H]
    \centering
    \begin{subfigure}{0.48\columnwidth}
        \centering
        \includegraphics[width=0.9\columnwidth]{figures/expantion fourie transform.png}
        \caption{inverse discrete fourie transform calculated from the interference pattern at of the helix }
        \label{fig:expansion inverse fourie transform measured}
    \end{subfigure}\hfill
    \begin{subfigure}{0.48\columnwidth}
        \centering
        \includegraphics[width=\columnwidth]{figures/expantion fourie transform magnified.png} % second figure itself
        \caption{magnification of\ref{fig:expansion inverse fourie transform measured}}
        \label{fig:expansion fourie transform magnified}
    \end{subfigure}

    \label{fig:expansion theory measurements}
\end{figure}

As can be seen in \ref{fig:expansion fourie transform magnified} The result shows similarity to the calculation done in \ref{fig:expansion theory measurements}
in shape as in the orientation of the spring to the orientation of the "X"